\title{cs805 Assignment 1}
\author{
  Shulang Lei\\
  200253624\\
  Department of Computer Science\\
  University of Regina\\
}
\date{\today}

\documentclass[12pt]{article}
\setlength{\parindent}{0in}
\usepackage{graphicx}
\usepackage{mathtools}
\usepackage{amsthm}
\usepackage{parskip}
\usepackage{hyperref}

\begin{document}
\maketitle

\begin{abstract}
  This assignment is written in literate programming style, generated by noweb, and rendered by LaTex.
\end{abstract}

\section{Question 1}
Let $n$ be a 3 tuple vector, and given that it is along $V1$. It is trivial that  we can imply:
\[
        n = \frac{V1}{[|V1|,|V1|,|V1|]}
\]
where $|V1| = \sqrt{V1_x^2+V1_y^2+V1_z^2}$

Thus n is now known.

By the definition of cross product, denoted as $\times$ here, knowning that $V1$ and $V2$ is non-collinear, we can also derive:
\[
        u = \frac{V2 \times V3}{[|V2 \times V3|,|V2 \times V3|,|V2 \times V3|]}
\]

Finally, it is also trivial that:
\[
        v = u \times n
\]

Thus, we now have u,v,n as:
\[
        n = \begin{bmatrix}
            \frac{V1_x}{\sqrt{V1_x^2+V1_y^2+V1_z^2}}
            &
            \frac{V1_y}{\sqrt{V1_x^2+V1_y^2+V1_z^2}}
            &
            \frac{V1_z}{\sqrt{V1_x^2+V1_y^2+V1_z^2}}
            \end{bmatrix}
\]
\[
        u = \begin{bmatrix}
            a
            &
            b
            &
            c 
            \end{bmatrix}
\]
\[
        v = \begin{bmatrix}
            a & b & c 
            \end{bmatrix}
\]

\end{document}
